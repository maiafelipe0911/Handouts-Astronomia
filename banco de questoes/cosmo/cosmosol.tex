\documentclass[11pt]{article}

\usepackage{maiacustom}

\begin{document}

\psettitle{Banco de questões de astronomia}

Estas questões foram produzidas/selecionadas cuidadosamente com o objetivo de preparar os estudantes para o processo seletivo de astronomia no Brasil. Algumas questões não são de autoria própria e estão devidamente sinalizadas por () antes do enunciado. O template do banco de questões é o mesmo do Professor \text{Kevin Zhou}. Seu trabalho é valioso, e diversas ideias desta lista podem ser encontradas em seus Handouts.

% \begin{psidea}{Título da Ideia}{}
% Ideia
% \end{psidea}

% \begin{psexample}{Título do Exemplo}{}
% Exemplo
% \end{psexample}

% \begin{pssolution*}{}{}
% Solução
% \end{pssolution*}

% \begin{psremark*}{Título da Observação}{}
% Observação
% \end{psremark*} 
\section{Cosmologia}

\pts{5}
\begin{pproblem}
    A Equação de Friedmann é uma das mais importantes para o estudo da cosmologia e do estudo sobre o universo. O objetivo do problema é deduzir as equações fundamentais da cosmologia. Primeiro, vamos com algumas definições:
    \begin{enumerate}[label=\roman*)]
        \item Devido à expansão do Universo, a distância entre dois pontos é dada por:
        \[r(t) = r_0 a(t)\]
        Onde \(a(t)\) é conhecido como o fator de escala e \(r_0\) é a distância medida em \(t = 0\). 
        
        \item O Universo segue a métrica de Robertson-Walker, a qual pode ser simplificada para:
        \[-c^2 dt^2 + dr^2 = 0\]
        \begin{alternativas}
            \item Encontre uma expressão para a distância comóvel, \(r_0\), na forma de integral, a partir das definições dadas anteriormente.
            \item A Primeira Equação de Friedmann tem forma:
            \[H(t)^2 \equiv \left(\frac{\dot{a}}{a}\right)^2 = k_1 \varepsilon(t) + \frac{C}{r_0^2 a^2}\]
            Encontre os valores das constantes \(k_1\) para um universo esférico, em função de constantes fundamentais. Na equação acima, \(\varepsilon(t)\) é a densidade de energia do universo, \(C\) é uma constante relacionada à sua energia e \(a\) é o fator de escala.

            \textbf{Dica:} Você pode obter essa equação tanto por conservação de energia ou utilizando a segunda lei de Newton.

            \item Repita o item anterior para um universo cilíndrico se expandindo radialmente. A equação encontrada deve ter forma: 

            \[\left(\frac{\dot{a}}{a}\right)^2 = f(a) \varepsilon(t) + \frac{C'}{r_0^2 a^2}\]

            Encontre a função \(f(a)\).

            \item Voltando agora para um "universo normal". Em cosmologia, definimos a densidade crítica de energia como sendo a densidade de energia de um universo em que \(C = 0\). Encontre uma expressão para a densidade crítica.
            \item Reescreva a primeira equação de Friedmann em função de:
            \[\Omega(t) = \frac{\varepsilon(t)}{\varepsilon_c(t)}\]
            Onde \(\varepsilon_c(t)\) é a densidade crítica de energia.
        \end{alternativas}
    \end{enumerate}
\begin{pssolution*}{}{ }
    \begin{alternativas}
        \item Isolando \(d_r\), 
        \[d_r = c dt\]

        Mas, por definição, \(r(t) = r_0 a(t) \rightarrow d_r = r_0 da\). Substituindo, 

        \[dr_0 = c\frac{dt}{a(t)}\]

        Integrando dos dois lados, 

        \[r_0 = c\int \frac{dt}{a(t)}\]

        \item \textbf{Método 1: Conservação de energia}
        Considere uma partícula de massa \(m\) localizada em um ponto do universo. A sua energia em relação ao centro do universo é dada por:

        \[E = \frac{mv^2}{2} - \frac{GM(r)m}{r}\]

        Onde \(v = \dot{r} = r_0 \dot{a}\) e \(M(r) = \frac{4}{3} \pi r^3 \rho(t)\). Dividindo a equação por \(m/2\) e aplicando as substituições, temos:

        \[\frac{2E}{m} = r_0^2 \dot{a}^2 - \frac{G}{r_0 a} \frac{8\pi (r_0 a)^3 \rho(t)}{3}\]

        Simplificando, 

        \[\frac{2E}{m} = r_0^2 \dot{a}^2 - \frac{8\pi G r_0^2 a^2 \rho(t)}{3}\]

        Multiplicando ambos os lados por \(\frac{1}{r_0^2 a^2}\), obtemos:

        \[\frac{2E}{mr_0^2 a^2} = \left(\frac{\dot{a}}{a}\right)^2 - \frac{8\pi G \rho(t)}{3}\]

        Reorganizando a equação e utilizando a equivalência massa-energia, \(m = E/c^2 \rightarrow \rho(t) = \varepsilon(t)/c^2\), temos:

        \[\left(\frac{\dot{a}}{a}\right)^2 = \frac{8\pi G}{3 c^2} \varepsilon(t) + \frac{2E}{mr_0^2 a^2}\]

        Da onde podemos concluir que,

        \[\boxed{k_1 = \frac{8\pi G}{3 c^2}}\]


        \textbf{Método 2: Segunda Lei de Newton}

        Escrevendo a segunda lei de Newton para um corpo com relação ao centro do universo,

        \[-\frac{GM(r)m}{r^2} = m \frac{d^2r}{dt^2}\]

        Note que, ao se expandir, o universo não cria massa, então a massa contida em uma camada \(r\) será constante ao longo do tempo, usando dessa propriedade, temos que:

        \[\frac{d^2r}{dt^2} = -\frac{GM(r)}{r^2}\]

        Multiplicando ambos os lados por \(dr/dt\),

        \[\frac{dr}{dt} \frac{d^2r}{dt^2} = -\frac{GM(r)}{r^2} \frac{dr}{dt}\]

        Reescrevendo,

        \[\dot{r} \frac{d\dot{r}}{dt} = -\frac{GM(r)}{r^2} \frac{dr}{dt}\]

        Multiplicando ambos os lados por \(dt\),

        \[\dot{r} d\dot{r} = -GM(r) \frac{dr}{r^2}\]

        Substituindo \(r = r_0 a\),

        \[r_0^2 \dot{a} d\dot{a} = -\frac{GM(r)}{r_0} \frac{da}{a^2}\]

        Integrando dos dois lados,

        \[r_0^2 \int \dot{a} da = -\frac{GM(r)}{r_0^2} \int \frac{da}{a^2}\]

        \[\frac{r_0^2 \dot{a}^2}{2} = \frac{GM(r)}{r_0 a} + A\]

        Onde \(A\) é uma constante de integração. Substituindo \(M(r)\),

        \[\frac{r_0^2 \dot{a}^2}{2} = \frac{4\pi G r_0^2 a^2}{3c^2} \varepsilon(t) + A\]

        Multiplicando a expressão por \(2/a^2 r_0^2\), temos:

        \[\left(\frac{\dot{a}}{a}\right)^2 = \frac{8\pi G}{3c^2} \varepsilon(t) + \frac{2A}{r_0^2 a^2}\]

        De onde podemos concluir, igualmente, que:

        \[\boxed{k_1 = \frac{8\pi G}{3c^2}}\]

        \item Utilizando a lei de Gauss,

        \[\oint \mathbf{g} \cdot d\mathbf{S} = -4\pi G M(r)\]

        Onde,

        \[\oint \mathbf{g} \cdot d\mathbf{S} = -g(2\pi r L)\]

        Para uma simetria cilíndrica. Substituindo na equação,

        \[g = \frac{2G M(r)}{r L}\]

        Utilizando o método da conservação de energia, temos primeiramente que calcular a energia potencial,

        \[U = -\int F dr = -\int \frac{2GM(r)m}{rL} dr = -\int \frac{2GM(r)m}{L} \frac{da}{a} - \frac{2GM(r)m}{L} \ln(a)\]

        Por conservação de energia,

        \[E = \frac{m r_0^2 \dot{a}^2}{2} - \frac{2GM(r)m}{L} \ln(a)\]

        Substituindo \(M(r) = \pi r_0^2 a^2 L \varepsilon(t)/c^2\),

        \[E = \frac{m r_0^2 \dot{a}^2}{2} - \frac{2\pi r_0^2 a^2 G m \ln(a)}{c^2} \varepsilon(t)\]

        Reorganizando,

        \[\left(\frac{\dot{a}}{a}\right)^2 = \frac{4\pi G \ln(a)}{c^2} \varepsilon(t) + \frac{2E}{mr_0^2 a^2}\]

        Da onde podemos obter,

        \[\boxed{f(a) = \frac{4\pi G}{c^2} \ln(a)}\]

        \item A equação de Friedmann para \(C = 0\) se reduz a

        \[H(t)^2 = \frac{8\pi G}{3c^2} \varepsilon_c(t)\]

        Isolando \(\varepsilon_c(t)\),

        \[\boxed{\varepsilon_c(t) = \frac{3c^2}{8\pi G} H^2(t)}\]

        Pela definição,

        \[\varepsilon(t) = \Omega (t) \varepsilon_c(t)\]

        Substituindo na equação,

        \[H^2(t) = \frac{8\pi G}{3c^2} \Omega(t) \varepsilon_c(t) + \frac{C}{r_0^2 a^2}\]

        Substituindo \(\varepsilon_c(t)\),

        \[H^2(t) = H^2(t) \Omega(t) + \frac{C}{r_0^2 a^2}\]

        Resolvendo para \(H^2(t)\),

        \[\boxed{H^2(t) = \frac{C}{r_0^2 a^2 (1 - \Omega(t))}}\]
    \end{alternativas}
\end{pssolution*}
\end{pproblem}


\pts{3}
\begin{pproblem}(Adaptado NAO 2019) 
    Considere um universo plano em que a constante gravitacional deixa de ser constante e passa a ser definida por
    \[G(a) = G_0 f(a)\]
    Onde \(f(a)\) é uma função do fator de escala.
    \begin{alternativas}
        \item Como seria a Equação de Friedmann nesse universo? Assuma que o universo é plano, \(C = 0\) e que ele é composto apenas por matéria bariônica ("clara"). Deixe sua resposta em função de \(H_0\), \(f(a)\), \(a\) e \(\Omega_{m,0}\), onde \(H_0\) é o valor da constante de Hubble no tempo atual, e \(\Omega_{m,0}\) é o parâmetro de densidade,
        
        No caso em que \(f(a) = e^{b(a-1)}\), onde \(b=2\).
        
        \item Estime a idade desse universo assumindo que ele é constituído apenas de matéria bariônica (matéria "clara").
                
        \item Qual o comportamento da idade quando \(t \rightarrow \infty\)?
    \end{alternativas}

    Talvez você ache as seguintes relações úteis:
    \begin{paracol}{2}
        \[\int_0^\infty x^2 e^{-x^2} = \frac{\sqrt{\pi}}{4}\]

        \switchcolumn

        \[\int_0^1 x^2 e^{-x^2} \approx 0,189\]
    \end{paracol}

\begin{pssolution*}{}{ }
    \begin{alternativas}
        \item Para um universo plano e contendo apenas matéria, a equação de Friedmann tem a forma:
        \[H(t)^2 = H_0^2 \Omega_m\]

        Onde \(\Omega_m = \frac{\rho_m(t)}{\rho_{c, 0}}\) e 

        \[\rho_c = \frac{3 H_0^2}{8\pi G}\]

        Para matéria bariônica,

        \[\rho_m(t) = \rho_0 a(t)^{-3}\]

        Então,

        \[\Omega_m = \Omega_{m,0} f(a) a^{-3}\]

        Para um universo composto apenas de matéria bariônica, temos \(\Omega_{m,0} = 1\). Assim, a equação de Friedmann fica:

        \[\boxed{H(a)^2 = H_0^2 \Omega_{m,0} f(a) a^{-3} \equiv H_0^2 f(a) a^{-3}}\]

        \item Usando que \(H = \frac{\dot{a}}{a}\), temos:

        \[t = \int dt = \int da \frac{dt}{da} = \int \frac{da}{\dot{a}}\]

        Multiplicando por \(a/a\),

        \[t = \int \frac{a}{\dot{a}a} da = \int \frac{da}{H(a) a}\]

        Substituindo o valor de \(H(a)\) encontrado no item anterior,

        \[t = \frac{1}{H_0^2} \int \frac{da}{\sqrt{f(a) a^{-3}}} = \frac{1}{H_0^2} \int a^{3/2} e^{-b(a-1)/2} da = \frac{e^{b/2}}{H_0^2} \int a^{3/2} e^{-b a/2} da\]
        
        Usando uma substituição da forma \(x = \sqrt{ba/2}\), temos:

        \[t = \frac{4\sqrt{2} e^{b/2}}{b^{3/2} H_0} \int x^2 e^{-x^2}\]

        Os limites de integração vão de \(a=0\) (início do universo) até \(a=1\) (momento atual). Como fizemos a substituição em \(x\),

        \[x = \sqrt{\frac{ab}{2}}\]

        \[t = \frac{4\sqrt{2} e^{b/2}}{b^{3/2} H_0} \int_0^{\sqrt{b/2}} x^2 e^{-x^2}\]

        Substituindo \(b=2\),

        \[t = \frac{4\sqrt{2} e}{2^{3/2} H_0} \int_0^1 x^2 e^{-x^2}\]

        Substituindo os valores e utilizando a integral fornecida, obtemos:

        \[\boxed{t \approx 15 \text{ Gyr}}\]

        O que é próximo do nosso universo!!

        \item Mudando os limites da integral para um tempo infinito, \(a(t) \rightarrow \infty\),

        \[\boxed{t_\infty = \frac{4\sqrt{2} e}{2^{3/2} H_0} \int_0^{\infty} x^2 e^{-x^2} = \frac{\sqrt{2\pi} e}{2^{3/2} H_0} \approx 34.1 \text{ Gyr}}\]

    \end{alternativas}
\end{pssolution*}
\end{pproblem}



\pts{5}
\begin{pproblem}(Lista 8 - 2021)
    A equação de Friedmann é dada por:
    \begin{equation}
        H^2 = \left( \frac{\dot{a}}{a} \right)^2 = \frac{8\pi G \varepsilon}{3c^2} - \frac{k c^2}{a^2},
    \end{equation}

    em que \(a\) é o fator de escala no tempo \(t\), \(\varepsilon\) é a densidade de energia no tempo \(t\) e \(k\) é o parâmetro que caracteriza a geometria do universo, podendo assumir qualquer valor real. Considerando um universo composto apenas por matéria bariônica não relativística e resolvendo essa equação diferencial não linear para \(k > 0\), obtêm-se as seguintes soluções em termos do parâmetro \(\theta \in [0, 2\pi]\):

    \begin{align}
        a(\theta) &= \frac{4\pi G \varepsilon_0}{3k c^4} \left( 1 - \cos\theta \right), \\
        t(\theta) &= \frac{4\pi G \varepsilon_0}{3k^{3/2}c^5} \left( \theta - \sin\theta \right).
    \end{align}

    Considere um universo com \(\Omega_0 = 4\) e \(H_0 = 67,4 \ \text{km/s/Mpc}\).

    \begin{alternativas}
        \item A partir da equação de Friedmann, mostre que \(k c^2 = H^2 \left( 1 - \Omega \right) a^2\). Por fim, reescreva as equações paramétricas de \(a\) e \(t\) em termos do parâmetro de densidade atual \(\Omega_0\) e da constante de Hubble atual \(H_0\), além do parâmetro \(\theta\). Não substitua os seus respectivos valores numéricos.
        
        \item Encontre a idade \(t_0\) do universo em questão em termos de \(H_0\) e em seguida em bilhões de anos.
        
        \item O chamado \textit{Lookback time}, \(\Delta t_L\), representa quanto tempo no passado o universo estava com certo fator de escala \(a\). Qual é \(\Delta t_L\) em bilhões de anos para quando o tamanho do universo era \(1/3\) do que é atualmente?
        
        \item Determine \(\theta_n\) e em seguida \(t_n\) para os quais \(H = 0\).
    \end{alternativas}

\begin{pssolution*}{}{}
    \begin{alternativas}
        \item Pela definição de $\Omega$

        \[\Omega = \frac{\varepsilon}{\varepsilon_c} = \frac{\frac{3c^2 H^2}{8\pi G} + \frac{3kc^4}{8\pi Ga^2}}{\frac{3c^2H^2}{8\pi G}} = 1 + \frac{kc^2}{H^2a^2}\]

        Isolando \(k\), 

        \[\boxed{kc^2 = H^2(\Omega -1)a^2}\]

        Assim como queríamos demonstrar. Para achar as equações de \(a\) e \(t\), vamos começar substituíndo \(kc^2\) e lembrando que no tempo atual, \(a=1\). Primeiro, vamos trabalhar somente em \(a(\theta)\).

        \[a(\theta) = \frac{4\pi G \varepsilon_0}{3c^2}\frac{(1-\cos\theta)}{H_0^2(\Omega_0-1)}\]

        Substituindo \(\varepsilon_0 = \Omega_0 \varepsilon_{c,0} = \Omega_0\frac{3H_0^2c^2}{8\pi G}\)

        Assim, 

        \[a(\theta) = \frac{4\pi G \Omega_0}{3c^2}\frac{3H_0^2c^2}{8\pi G}\frac{(1-\cos\theta)}{H_0^2(\Omega_0-1)} = \frac{\Omega_0(1-\cos\theta)}{2(\Omega_0-1)}\]

        \[\boxed{a(\theta) = \frac{\Omega_0(1-\cos\theta)}{2(\Omega_0-1)}}\]
        
        Fazendo a mesma coisa para \(t(\theta)\) obtemos: 

        \[t(\theta) = \frac{4\pi G\varepsilon_0 }{3c^2}\frac{(\theta-\sin\theta)}{(kc^2)^{3/2}} = \frac{4\pi G \varepsilon_0}{3c^2H_0^3(\Omega_0-1)^{3/2}}(\theta-\sin\theta)\]

        Substituindo \(\varepsilon_0\), 

        \[t(\theta) = \frac{4\pi  G\Omega_0}{3c^2H_0^3(\Omega_0-1)^{3/2}}(\theta-\sin\theta)\frac{3H_0^2c^2}{8\pi G}\]

        \[\boxed{t(\theta) = \frac{\Omega_0(\theta-\sin\theta)}{2H_0(\Omega_0-1)^{3/2}}}\]

        
        \item Atualmente, \(a(\theta)=1\), assim, 
        
        \[2(\Omega_0-1) = \Omega_0(1-\cos\theta) \rightarrow \cos\theta = 1 -\frac{2(\Omega_0-1)}{\Omega_0} = -0,5\]
        \[\theta = \frac{2\pi}{3} \text{ ou }\frac{4\pi}{3}\]

        Substituindo esse valor na expressão do tempo, 

        \[t\left(\frac{2\pi}{3}\right) = \frac{0,4728}{H_0} \ , \ \ t\left(\frac{4\pi}{3}\right) = \frac{1,9456}{H_0}\]

        Porém, o nosso problema não admite essas duas soluções, uma vez que em um universo fecahdo o mesmo expande até um determinado tamanho e após isso começa a contrair, porém, como \(H_0>0\), o universo está atualmente expandindo, o que nos leva a crer que a idade atual do universo é o menor valor entre os dois, assim, a idade do universo é dada por 

        \[\boxed{t\left(\frac{2\pi}{3}\right)\approx 6,9 \text{ bilhões de anos}}\]

        \item Substituindo \(a=1/3\) e fazendo o mesmo processo que no item anterior, 
        
        \[\frac{1}{3} = \frac{\Omega_0(1-\cos\theta)}{2(\Omega-1)}\]

        Resolvendo para \(\theta\), obtemos \(\theta = \pi/3 \) ou \(5\pi/3\). Como o tempo de Lookback representa um tempo passado, \(\theta\) precisa ser menor do que \(2\pi/3\) (que representa o ponto atual). Desse modo, \(\theta = \pi/3\). Substituindo na fórmula do tmepo, 

        \[t(\pi/3) = \frac{0,0697}{H_0} \approx 1,03 \text{ bilhões de anos}\]

        Assim, \(\boxed{\Delta t_L = t(2\pi/3) - t(\pi/3) = 5,87 \text{ bilhões de anos.}}\)

        \item Pela definição de \(H\), temos, \(H = \frac{1}{a}\frac{da}{dt}\). Usando a regra da cadeita, 
        
        \[H = \frac{1}{a}\frac{d a}{d\theta}\left(\frac{dt}{d\theta}\right)^{-1}\]

        Derivando separadamente, 

        \[\frac{da}{d\theta} = \frac{d}{d\theta}\frac{\Omega_0(1-\cos\theta)}{2(\Omega_0-1)} = \frac{\Omega_0\sin\theta}{2(\Omega_0-1)}\]

        \[\frac{dt}{d\theta} = \frac{d}{d\theta}\frac{\Omega_0(\theta-\sin\theta)}{2H_0(\Omega_0-1)^{3/2}} = \frac{\Omega_0(1-\cos\theta)}{2H_0(\Omega_0-1)^{3/2}}\]

        Agora, substituindo na expressão de \(H\), 

        \[H = \frac{2(\Omega_0-1)}{\Omega_0(1-\cos\theta)}\frac{\Omega_0\sin\theta}{2(\Omega_0-1)}\frac{2H_0(\Omega_0-1)^{3/2}}{\Omega_0(1-\cos\theta)}\]
        \[H = \frac{2H_0(\Omega_0-1)^{3/2}\sin\theta}{\Omega_0(1-\cos\theta)^2}\]

        Para \(H=0\), temos que \(\sin\theta = 0\), mas \(\cos\theta \ne 1\). Nessas condições, o único valor que atende é \(\boxed{\theta = \pi}\). Substituindo na equação para o tempo, 

        \[\boxed{t(\pi) = \frac{\pi\Omega_0}{2H_0(\Omega_0-1)^{3/2}}= \frac{1,209}{H_0} \approx 17,86 \text{ bilhões de anos}}\]

    \end{alternativas}
\end{pssolution*}
\end{pproblem}

\end{document}